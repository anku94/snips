\documentclass[sigconf,authordraft,natbib=false]{acm-template/acmart}

% -- we use biblatex instead of latex --
\usepackage[datamodel=acm-template/acmdatamodel, style=acm-template/acmnumeric, backend=biber]{biblatex}
% -- for code blocks --
\usepackage{listings}
\usepackage{xcolor}
\usepackage{algorithm}
\usepackage{algpseudocodex}
\usepackage{cleveref}

\usepackage{caption}
\usepackage{subcaption}

% define a command for red text: \red{abc} will print abc in red
\newcommand{\red}[1]{\textcolor{red}{#1}}

\newcommand{\comp}{\texttt{compute}}
\newcommand{\sync}{\texttt{sync}}

\addbibresource{bib/amr.bib}
\addbibresource{bib/netsketch.bib}

\definecolor{mauve}{rgb}{0.58, 0.44, 0.86}  % Define mauve color

\lstdefinelanguage{customcpp}{
  language=C++,
  morekeywords={foreach}
}

\lstdefinestyle{mystyle}{
    % backgroundcolor=\color{lightgray},   
    commentstyle=\color{olive},
    keywordstyle=\color{blue},
    numberstyle=\tiny\color{gray},
    stringstyle=\color{mauve},
    basicstyle=\ttfamily\footnotesize,
    breakatwhitespace=false,         
    breaklines=true,                 
    captionpos=b,                    
    keepspaces=true,                 
    numbers=left,                    
    numbersep=5pt,                  
    showspaces=false,                
    showstringspaces=false,
    showtabs=false,                  
    tabsize=2,
    lineskip=-1ex
}
\lstset{style=mystyle}

\makeatletter
\newcommand{\guideline}[2]{
  \vspace{0.6em}\noindent \textbf{Guideline #1: } \emph{#2}\vspace{0.3em}
  \par\@afterindentfalse\@afterheading
}
\makeatother

% ----
\begin{document}

\title{AMR Workload Analysis and Profiling}

\author{CMU}

%%
%% The abstract is a short summary of the work to be presented in the
%% article.
\begin{abstract}
  A clear and well-documented \LaTeX\ document is presented as an
  article formatted for publication by ACM in a conference proceedings
  or journal publication. Based on the ``acmart'' document class, this
  article presents and explains many of the common variations, as well
  as many of the formatting elements an author may use in the
  preparation of the documentation of their work.
\end{abstract}


\keywords{Do, Not, Us, This, Code, Put, the, Correct, Terms, for,
  Your, Paper}

\maketitle

\input{tex/intro}
\input{tex/background}
\input{tex/model}
\input{tex/guidelines}
\input{tex/design}
\input{tex/concl}

\red{ --- notes follow --- }

\section{References}

\printbibliography


\end{document}
